\documentclass[12pt, a4paper]{article}
\usepackage[utf8]{inputenc}
\usepackage[brazilian]{babel} % Hifenização e dicionário
\usepackage[left=3.00cm, right=2.00cm, top=3.00cm, bottom=2.00cm]{geometry}
\usepackage{enumitem} % Para itemsep etc
\usepackage{longtable} % Dependência do longtabu
\usepackage{tabu} % Para melhor criação de tabelas
\usepackage{listings} % Para códigos
\usepackage{lstautogobble} % Códigos indentados corretamente
\usepackage{color} % Para coloração de códigos
\usepackage{libertine} % Linux Libertine
\usepackage{parskip} % Linha em branco entre parágrafos em vez de recuo
\usepackage{graphicx}
\usepackage{verbatim} % Para comentários
\usepackage{amssymb} % Para \blacktriangle e \blacktriangledown
\usepackage[breaklinks]{hyperref}

\newcommand{\code}[1]{\textbf{\lstinline{#1}}}

\usepackage{listings}
\lstset{
    autogobble,
    columns=fullflexible,
    showspaces=false,
    showtabs=false,
    breaklines=true,
    showstringspaces=false,
    breakatwhitespace=true,
    escapeinside={(*@}{@*)},
    basicstyle=\ttfamily\footnotesize,
    frame=l,
    framesep=12pt,
    xleftmargin=12pt,
    tabsize=4,
    captionpos=b
}

\begin{document}

\begin{titlepage}
\begin{center}
    \textsc{INPE - Instituto Nacional de Pesquisas Espaciais}

    \vfill

    Interface gráfica para Estação Multi-Missão de Natal \\
    Manual de referência para desenvolvedores

    \vfill

    \begin{flushright}
    Desenvolvido por Felipe Cortez de Sá sob supervisão de José Marcelo e Lúcio Jotha
    \end{flushright}

    \vfill
    Natal, 2017
\end{center}
\end{titlepage}

\tableofcontents

\newpage

\section{Classes}
\subsection{Interface}
\subsubsection{AddTrackerDialog}
Referente à caixa de diálogo

\subsubsection{JoystickWidget}
Referente ao joystick virtual

\subsubsection{MainWindow}
A janela principal do programa. Responsável por conectar diversas partes da
interface juntas.

\subsubsection{ManualControlDialog}
Referente à caixa de diálogo do controle manual, englobando tanto o joystick
quanto o formulário de posicionamento

\subsubsection{NextPassesView}
Referente à visualização de próximas passagens visível na posição inferior do
programa. Contém lógica de desenho.

\subsubsection{SettingsDialog}
Referente à caixa de diálogo com opções de configuração. As mudanças só são efetivadas ao clicar em aplicar.

\subsection{Dados}
\subsubsection{Efem}
Não utilizado?

\subsubsection{Tracker}
Contém dados de um satélite rastreado

\subsubsection{TrackerListModel}
Armazena a lista de satélites rastreados em um formato compatível com o Qt
necessário para ser visualizado numa lista.

\subsection{Lógica e miscelânea}
\subsubsection{Control}
Contém a lógica de movimentação da antena, bem como a programação de passagens
e conexão com a porta serial.

\subsubsection{Helpers}
Contém funções independentes de classe (estáticas) que ajudam em diversas
partes do programa.

\subsubsection{Network}
Contém funções para conexão com a Internet.

\subsubsection{Serial}
Biblioteca para conexão serial.

\end{document}
