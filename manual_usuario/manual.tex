\documentclass[12pt, a4paper]{article}
\usepackage[utf8]{inputenc}
\usepackage[brazilian]{babel} % Hifenização e dicionário
\usepackage[left=3.00cm, right=2.00cm, top=3.00cm, bottom=2.00cm]{geometry}
\usepackage{enumitem} % Para itemsep etc
\usepackage{longtable} % Dependência do longtabu
\usepackage{tabu} % Para melhor criação de tabelas
\usepackage{listings} % Para códigos
\usepackage{lstautogobble} % Códigos indentados corretamente
\usepackage{color} % Para coloração de códigos
\usepackage{libertine} % Linux Libertine
\usepackage{parskip} % Linha em branco entre parágrafos em vez de recuo
\usepackage{graphicx}
\usepackage{verbatim} % Para comentários
\usepackage[os=win]{menukeys} % Para botões, diretórios e teclas
% http://repositorios.cpai.unb.br/ctan/macros/latex/contrib/menukeys/menukeys.pdf
\usepackage{amssymb} % Para \blacktriangle e \blacktriangledown
\usepackage[breaklinks]{hyperref}

\renewmenumacro{\menu}[>]{angularmenus}
\renewmenumacro{\keys}[+]{shadowedangularkeys}

\begin{document}

\begin{titlepage}
\begin{center}
    \textsc{INPE - Instituto Nacional de Pesquisas Espaciais}

    \vfill

    Interface gráfica para Estação Multi-Missão de Natal \\
    Manual de referência para Usuários

    \vfill

    \begin{flushright}
    Desenvolvido por Felipe Cortez de Sá sob supervisão de José Marcelo e Lúcio Jotha
    \end{flushright}

    \vfill
    Natal, 2017
\end{center}
\end{titlepage}

\tableofcontents

\newpage

% atalho \keys{\ctrl + \shift + F}
\section{Adicionando, editando e removendo satélites}
É possível adicionar um novo satélite clicando em \menu{Adicionar novo
satélite}. Uma caixa de diálogo aparecerá, e um dos \emph{two-line element
sets} disponíveis em \href{https://www.celestrak.com/NORAD/elements/resource.txt}{Celestrak}
deve ser copiado e colado.

Clicando num satélite na lista de satélites, é possível removê-lo ou alterar
sua prioridade, e a caixa abaixo da lista de satélites adicionados mostra
informações de azimute, elevação e próxima passagem para o satélite
selecionado.

\subsection{Definindo prioridades}
Selecionado um satélite, \menu{$ \blacktriangle $} eleva sua prioridade e
\menu{$ \blacktriangledown $} diminui. Isso significa que caso duas passagens
ocorram no mesmo período, o satélite com maior prioridade será o rastreado pela
antena.

% \menu{\return}
\section{Configurando porta USB}
É possível escolher a porta USB para comunicação com dispositivo Arduino no
menu \menu{Arquivo > Configurações}.

\section{Controle manual}
É possível movimentar a antena independentemente da lista de passagens através
de \menu{Arquivo > Controle manual}.

\subsection{Joystick}
O joystick é controlado pelo mouse clicando e arrastando o círculo menor.
Movimentar o joystick para a esquerda e direita diminui e aumenta o ângulo de
azimute, respectivamente, e movimentar para baixo e cima altera a elevação da antena.

\subsection{Posição}
É possível mandar a antena para uma posição específica na aba de posição.
Pode-se mandar azimute ou elevação desejados individualmente com botões de
\menu{\textgreater} ou escolher uma posição específica e clicar em \menu{Mandar ambos}.

\section{Lista de passagens}
\subsection{Passagens inversas}
Quando detecta-se que uma passagem faz a antena passar pelo fim de curso, os
valores enviados para a antena são invertidos e a passagem é especificada como
inversa na tabela.

\section{Visualização gráfica de próximas passagens}
É possível ver todas as passagens para os satélites cadastrados nas próximas 7
horas. Cada retângulo representa uma passagem, com o tempo no eixo x, e a parte
esquerda do retângulo representando a aquisição de sinal e a parte direita a
perda de sinal.

\end{document}
